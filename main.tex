 \documentclass[a4paper,openright,12pt]{report}
\usepackage{graphicx}
\usepackage[utf8]{inputenc}
\usepackage[hidelinks]{hyperref} 
\usepackage[hidelinks]{hyperref} 
\usepackage{multirow}
\usepackage{tikz}
\usepackage{fullpage}
\usepackage{parskip}
\usepackage{subfiles}
\usepackage{graphicx}
\usepackage{url}
\usepackage{xspace}
\usepackage{cite}\usepackage{cancel}
\usepackage{caption}
\usepackage{mathtools}
\usepackage{array}
\usepackage{listofsymbols} 
\usepackage[spanish, es-tabla, es-lcroman]{babel}
\usepackage{float}
\usepackage{longtable} 
\usepackage{nomencl}
\makenomenclature
\usepackage[export]{adjustbox}

\renewcommand{\nomname}{Glosario}
\newcommand*\circled[1]{\tikz[baseline=(char.base)]{
            \node[shape=circle,draw,inner sep=0.7pt] (char) {#1};}}
	
\hyphenation{op-tical net-works semi-conduc-tor cons-te-lla-tion co-nnecti-vi-ty}


\setlength{\parindent}{50pt}
\graphicspath{ {D:/Thesis/Escrito/img/} }
\newtheorem{q}{Question}[section]


\captionsetup{position=below}
\newtheorem{remarque}{Remarque}

\newcommand{\R}{\mathcal{R}}
\newcommand{\N}{\mathcal{N}}

\newcommand{\var}[1]{{\small\texttt{#1}} }
\newcommand{\lcavol}{\var{LCAVOL}}
\newcommand{\lweight}{\var{LWEIGHT}}
\newcommand{\age}{\var{AGE}}
\newcommand{\lbph}{\var{LBPH}}
\newcommand{\svi}{\var{SVI}}
% \newcommand{\lcp}{\var{LCP}}
\newcommand{\gleason}{\var{GLEASON}}
\newcommand{\pgg}{\var{PGG45}}
\newcommand{\lpsa}{\var{LPSA}}
\newcommand{\grad}{$^{\circ}$}
\usepackage{listings}
\setcounter{tocdepth}{4}
\setcounter{secnumdepth}{4}
\usepackage{inconsolata}
\usepackage{xcolor}
\hypersetup{
    colorlinks,
    linkcolor={red!0!black},
    citecolor={black},
    urlcolor={blue!100!black}
}

\newcommand{\todo}[1]{\textcolor{blue}{TODO: #1}}

\usepackage[T1]{fontenc}   
\usepackage[utf8]{inputenc}

\fboxsep4mm

\fboxrule1.0pt
\usepackage{fancybox}
\usepackage{xcolor}

\definecolor{Zgris}{rgb}{0.87,0.85,0.85}

\newsavebox{\BBbox}
\newenvironment{DDbox}[1]{
\begin{lrbox}{\BBbox}\begin{minipage}{\linewidth}}
{\end{minipage}\end{lrbox}\colorbox{Zgris}{\usebox{\BBbox}} \\
[.5cm]}
\title{Proyecto Integrador Dardo Ariel Viñas Viscardi}
\begin{document}

%----------------------------------------------------------------------------------------
%	TITLE PAGE
%----------------------------------------------------------------------------------------

\begin{titlepage}
\begin{center}

\textsc{Universidad Nacional de Córdoba}\\[1.5cm] % University name

\includegraphics[width=0.3\textwidth]{img/logo_unc.pdf}~\\[1cm]

\textsc{\Large Proyecto Integrador Ingeniería en Computación }\\[0.5cm] % Thesis type

\rule{1.5\textwidth}{.4pt} % Horizontal line
\title{Title }\\[0.4cm] % Thesis title 
\textbf{\Huge{NetFi2}}
\rule{1.5\textwidth}{.4pt} %% Horizontal line
 
\begin{minipage}{0.4\textwidth}
\begin{flushleft} \large
\emph{Autor: Viñas Viscardi Dardo Ariel}\\


\end{flushleft}
\end{minipage}
\begin{minipage}{0.4\textwidth}
\begin{flushright} \large
\emph{Director: Dr. Ferreyra Pablo} \\
\emph{Codirector: Dr. Juan Fraire} \\
 
\end{flushright}
\end{minipage}\\[1cm]

\begin{table}[H]
\centering
\label{my-label}

\end{table}
 % University requirement text
%\textit{ }\\[0.4cm]

\large \textit{Laboratorio de Circuitos y Sistemas Robustos (LCSR)}\\ Facultad de Ciencias exactas Físicas y Naturales
\large \textit{}\\
\large \textit{}\\
\large \textit{}\\
\large \textit{ Lab. Techniques de l'Informatique et de la Microélectronique pour l'Architecture des systèmes intégrés (TIMA)}\\ Instituto Politécnico de Grenoble, Francia
 \large \textit{}\\
  \large \textit{}\\
   \large \textit{}\\
{\large \today}\\[2cm] % Date
%\includegraphics{Logo} % University/department logo - uncomment to place it
 
\vfill
\end{center}
\end{titlepage}


$\ $
\thispagestyle{empty} % para que no se numere esta pagina
% 


 % para comenzar la numeracion de paginas en números romanos

\chapter*{Resumen} % si no queremos que añada la palabra "capítulo"
\addcontentsline{toc}{chapter}{Resumen} % si queremos que aparezca en el índice
\markboth{RESUMEN}{RESUMEN} % encabezado
La microelectrónica constituye un elemento clave en las innovaciones de la sociedad moderna. En efecto, la industria busca reducir el tamaño y consumo de los dispositivos electrónicos para mejorar su rendimiento. Sin embargo, las escalas de integración alcanzadas hacen que los circuitos integrados resulten cada vez más sensibles a los efectos de la radiación. Con el fin de evaluar la susceptibilidad de estos sistemas digitales, se han propuesto diversas metodologías y herramientas. En este trabajo retomamos una metodología previa de inyección por emulación para extenderla, actualizarla, y mejorarla bajo el nombre de NETFI-2. A lo largo del texto describimos en detalle esta nueva estrategia para luego proponer un caso de uso práctico basado en un circuito existente de una máquina bayesiana. Por un lado, este experimento nos permite obtener un primer análisis sobre la sensibilidad a los efectos de la radiación de este tipo de circuitos, y por otro, contrastar los resultados proporcionados por NETFI-2 con metodologías de simulación análogas. Luego de realizar estas pruebas demostramos que ocupando sólo una pequeña área del dispositivo destino, NETFI-2 proporciona resultados coherentes y realistas mejorando en más de un 300\% el tiempo de ejecución de la campaña de inyección de fallas respecto a la simulación. En el marco de este proyecto se ha realizado 4 publicaciones, las cuales son los frutos de esta investigación y las cuales podrán resumir de igual manera todo el trabajo realizado \cite{SanLuis} \cite{Alemania} \cite{Alemania2} \cite{Bariloche}.

Un aspecto a remarcar de este trabajo es que forma parte del primer proyecto de investigación y desarrollo que se está realizando en el LCSR (Laboratorio de Circuitos y Sistemas Robustos), recientemente creado en nuestra Facultad.

%\chapter*{Introducción} % si no queremos que añada la palabra "capítulo"
%\addcontentsline{toc}{chapter}{Introducción} % si queremos que aparezca en el índice
%\markboth{Introducción}{Introducción} % encabezado
		


\cleardoublepage
\addcontentsline{toc}{chapter}{Lista de figuras} % para que aparezca en el índice de contenidos
% * <vinas.dardoariel@gmail.com> 2016-09-21T14:52:00.680Z:
%
% ^.
\listoffigures % indice de figuras

\cleardoublepage
\addcontentsline{toc}{chapter}{Lista de tablas} % para que aparezca en el indice de contenidos
\listoftables % indice de tablas

\tableofcontents % indice de contenidos
% \cleardoublepage

\addcontentsline{toc}{chapter}{Glosario}

\nomenclature{DUT}{Device Under \textit{test}} 
\nomenclature{FPGA}{Field-Programmable Gate Array } 
\nomenclature{TIMA}{Techniques de l'Informatique et de la Microélectronique pour l'Architecture des systèmes intégrés}
\nomenclature{SET}{Single Event Transient}
\nomenclature{SEU}{Single Event Upset}
\nomenclature{MBU}{Multi Bit Upset}
\nomenclature{MCU}{Multi Cell Upset}
\nomenclature{ESA}{European Space Agency}
\nomenclature{AXI}{Advanced eXtensible Interface}
\nomenclature{AMBA}{Advanced Microcontroller Bus Architecture}
\nomenclature{UART}{Universal Asynchronous Receiver Transmitter}
\nomenclature{VHDL}{VHSIC Hardware Description Language}
\nomenclature{RTL}{Register-Transfer Level}
\nomenclature{IDE}{Integrated Development Environment}
\nomenclature{JTAG}{Joint \textit{test} Action Group}
\nomenclature{LUT}{Lookup Table}
\nomenclature{FF}{Flip Flop}
\nomenclature{RIS}{Robust Integrated Systems}
\nomenclature{NETFI2}{Netlist Fault Injection 2}
\nomenclature{MODNET}{MODify NETlist}
\nomenclature{LET}{Lineal Energy Tranfer}
\nomenclature{NIEL}{Non Ionising Energy Loss}
\nomenclature{SEE}{Single Event Effect}
\nomenclature{SEL}{Single Event Latchup}
\nomenclature{EDIF }{Electronic Design Interchange Format}
\nomenclature{GeV}{Giga Electrón Volt}
\nomenclature{MeV}{Mega Electrón Volt}
\nomenclature{}{}


\printnomenclature

% \newpage


\pagenumbering{arabic} 

\subfile{content/chapters/01/chap01.tex}
\subfile{content/chapters/02/chap02.tex}
\subfile{content/chapters/03/chap03.tex}
\subfile{content/chapters/04/chap04.tex}
\subfile{content/chapters/05/chap05.tex}
\subfile{content/chapters/06/chap06.tex}



    
\bibliographystyle{content/bibtex/bst/unsrt}


\bibliography{content/bibtex/bib/bibliography}
Toda la bibliográfica consultada en sitios webs corresponde a consultas realizadas entre los meses Marzo y Octubre del año 2017. 

% \appendix

% \subfile{content/chapters/apend/01/apend01.tex}
% \subfile{content/chapters/apend/02/apend02.tex}
% \subfile{content/chapters/apend/03/apend03.tex}

\end{document}

\tableofcontents
\listoffigures
\listoftables
