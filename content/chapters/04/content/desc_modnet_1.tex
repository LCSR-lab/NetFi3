Synplify Premier cuenta con una funcionalidad "shell" como ellos la describen, que significa básicamente que el software puede correrse de la terminal sin la necesidad de una GUI (esto nos permite automatizar el proceso de síntesis).

Para obtener la NetList de un diseño RTL se necesitan 2 componentes: el primero, es un script que se ejecuta de forma remota y realiza la síntesis en si. El segundo es un script "tcl", que no es otra cosa que un lenguaje que Sinplify puede interpretar para realizar diferentes tareas. En el caso de la BM Slice, este script tiene la siguiente forma: \break


\begin{lstlisting}
    # Crea un proyecto nuevo en Sinplify
    project -new proyectos/bm_slice.prj

    # Guarda el proyecto
    project -save proyectos/bm_slice.prj

    # Agrega los componentes de la BM Slice 
    add_file -vhdl ./bm.vhd
    add_file -vhdl ./bm_slice.vhd
    add_file -vhdl ./params.vhd
    add_file -vhdl ./sto_add2.vhd
    add_file -vhdl ./sto_add3.vhd
    add_file -vhdl ./sto_mul2.vhd
    add_file -vhdl ./sto_mul3.vhd

    # Guardar netlist en formato verilog 
    set_option -result_file proyectos/rev_1/bm_slice.vm

    # Configuracion para Nexys 4
    set_option -technology ARTIX7
    set_option -part XC7A100T
    set_option -package CSG324
    set_option -grade -1

    # Comienzo de la sintesis
    project -run
    exit    
    
\end{lstlisting}

La salida de Synplify en \path{proyectos/rev_1/bm_slice.vm} es la NetList que necesitamos como entrada a MODNET v2