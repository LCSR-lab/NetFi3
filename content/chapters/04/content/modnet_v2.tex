MODify NETlist (MODNET) es una herramienta desarrollada por Wassim Mansou para su tesis doctoral en el laboratorio TIMA. El software fue diseñado para tomar una Netlist de un circuito a  someter, realizándole cambios en los lugares que se consideran sensibles, para que la inyección de fallos sea posible. 

Uno de los problemas más grandes de este software es su falta de capacidad para ser automatizable dentro de un flujo completo de NetFI. Además de esto, MODNET se desarrollo en lenguaje C\#, el cual, para ser compilado depende por completo de la herramienta Visual Studio 2012, la cual es privativa y de uso bajo licencia, convirtiendo al software en uno de los componentes más problematicos del sistema.

En este trabajo se tomó la funcionalidad central de MODNET y se realizó una ingeniería inversa de su código para obtener así las claves necesarias para implementarla en cualquier lenguaje. El objetivo con esto no es solo el poder hacer cambios grandes en la arquitectura general del funcionamiento del software en si, pero además para poder documentar extensivamente sus funcionalidades y de esta manera poder extender en el futuro sus capacidades para poder realizar inyecciones en cualquier tipo de Netlist (desde una de Vivado, hasta una del software open-source de síntesis de FPGA YOSIS)