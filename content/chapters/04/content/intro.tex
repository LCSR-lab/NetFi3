NetFi-3 es una actualización de un trabajo previo~\cite{netfi2} el cual, a partir de un re-diseño original de NetFi~\cite{6555963} se obtuvo una metodología con la cual se podía realizar inyección de fallos a nivel RTL. Dicho método si bien demostró ser eficiente a nivel hardware (puesto que ya no era necesario de hardware adicional a la FPGA donde se realizaban las pruebas de inyección), dejaba demasiados cabos sueltos en su contraparte de software, debido a la inmensa curva de aprendisaje requerida para implementar siquiera el diseño más simple bajo NetFi-2.

Este trabajo entonces logra convertir lo que una vez era una metodología, a lo que es ahora, un completo suite de herramientas automatizadas que nos permiten no solo realizar inyecciones de forma eficiente, rápida y con la seguridad requerida al momento de trabajar con diseños RTL, sino también nos brinda un total y completo control sobre como realizar dichas inyecciones, obtención de resultados. Por último, pero no menos importante, las herramientas ahora implementadas cumplen con la ventaja de ser extensibles de una manera mucho más fácil, no solo por la extensa y precisa documentación elaborada, sino por la descentralización de componentes reutilizables de software implementados.

El núcleo central y más importante de este trabajo es la re arquitectura realizada a MODNET, su migración a python y nuevo caracter de implementación, el cual nos permite automatizar un flujo de inyección de fallos con una simple CLI (command line interface)