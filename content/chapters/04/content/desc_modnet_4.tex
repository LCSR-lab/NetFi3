MODNET v2 cuenta con una CLI (command line interface), a diferencia de su predecesor. La herramienta se encuentra publicada en PyPI, por lo tanto su instalación es muy sencilla, simplemente corriendo:

\break

\begin{lstlisting}
    pip install mod-net
\end{lstlisting}

Para su utilización, y en su estado actual, MODNET necesita como entrada

\begin{itemize}
    \item Ubicación de la NetList a procesar,
    \item Nombre del módulo top del sistema,
    \item Directorio de salida (por defecto utiliza \path{ubicacion_de_la_netlist/output})
\end{itemize}

Entonces se corre el comando \emph{modnet} con sus parámetros:

\break

\begin{lstlisting}
    modnet --netlist ejemplo/netlist.vm --top-module top_module --outdir ./output
\end{lstlisting}

De esta manera MODNET creará un árbol de archivos con la salida correspondiente al siguiente esquema:

\begin{forest}
    for tree={
      font=\ttfamily,
      grow'=0,
      child anchor=west,
      parent anchor=south,
      anchor=west,
      calign=first,
      edge path={
        \noexpand\path [draw, \forestoption{edge}]
        (!u.south west) +(7.5pt,0) |- node[fill,inner sep=1.25pt] {} (.child anchor)\forestoption{edge label};
      },
      before typesetting nodes={
        if n=1
          {insert before={[,phantom]}}
          {}
      },
      fit=band,
      before computing xy={l=15pt},
    }
  [example
    [netlist.vm]
    [output
        [modulo1\_con\_iny.v]
        [modulo2\_con\_iny.v]
        [modulo3\_con\_iny.v]
    ]
    [src
        [modulo1.v]
        [modulo2.v]
    ]
  ]
  \end{forest}