Este proyecto integrador se realizó con el grupo de investigación  \textit{Robust Integrated Systems} (RIS) del laboratorio  \textit{Techniques de l'Informatique et de la Microélectronique pour l'Architecture des systèmes intégrés} (TIMA), en la ciudad de Grenoble, Francia. El laboratorio TIMA aportó todos los conocimientos y recursos necesarios que permitieron la realización de este trabajo.

TIMA es un laboratorio de investigación ubicado en  Grenoble, Francia. La ciudad porta el título de  segundo polo de Micro y Nano Electrónica más importante de Europa. Cuenta con amplia gama de servicios, estructuras, empresas y laboratorios dedicados principalmente a estas áreas. Entre las más notables cabe destacar los dos aceleradores de partículas y el polo tecnológico.

El laboratorio TIMA  está bajo la tutela del Centro Nacional de Investigación Científica (CNRS), del Instituto Politécnico de Grenoble (Grenoble INP) y de la Universidad Grenoble Alpes (UGA). 

Las investigaciones en el laboratorio TIMA están divididas en 5 equipos:
\begin{itemize}
\item \textit{System Level Synthesis} ~\cite{SLS}
\item \textit{Architectures and Methods for Resilient Systems} ~\cite{AMfoRS}
\item \textit{Reliable Mixed-signal Systems} ~\cite{RMS}
\item \textit{Design of Integrated devices}, Circuits and Systems ~\cite{CDSI}
\item \textit{Robust Integrated Systems} ~\cite{RIS}
\end{itemize}