Desde su primera aparición, NetFI ha intentado resolver los problemas presenten a los sistemas comtemporáneos de sistemas de inyección de fallas en niveles RTL. Ya sea, reduciendo los costos de la FPGA necesaria para realizar una camapaña de inyecciones o comprimiendo la metodología a caber dentro de una sola FPGA, no solo reduciendo costos pero también tiempo de implementación.

Esto último fue para NetFI-2 uno de los componentes claves en su desarrollo. La reducción en tiempo de implementación fue conseguida, pero sobretodo la reducción de costos. Siendo objetivo, someter a un DUT bajo la metodología NetFI-2 fue singularmente doloroso. Al incursionar en el experimento nos encontraremos, desde falta de documentación a un sin fin de pasos manuales que se deden realizar en un cierto órden sin errores para la implementación de la metodología. Esto no es solo propenso a errores, sino que significa un esfuerzo a veces mayor sobre la metodología de inyecciones, que el esfuerzo que llevó a desarrollar el DUT sometido. Y para finalizar, NetFI-2 no permite una aproximación real a circuitos con LUTs de mayor tamaño, ya que solo niega la salida, podría haber en un chip real muchísimos puntos sensibles dentro de la lógica encerrada dentro de una LUT6, por ejemplo.

Este trabajo por ende pretende atacar esto. ¿Podemos convertir a la metodología desarrollada en TIMA, en un sistema completamente automático? ¿Es posible realizar inyección de fallas en mi dispositivo sin necesariamente saber como el sistema las realiza? Y para finalizar, ¿podemos aumentar la granularidad de la inyeción de fallas?
