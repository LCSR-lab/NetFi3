El trabajo inicio su desarrollo en el laboratorio TIMA, el cual aportó todos los conocimientos, recursos físicos  y materiales. Luego se completó dentro del laboratorio LCSR, para darle los toques finales necesarios para automatizar por completo la herramienta.

La idea  tomó como punto de referencia los métodos de inyeccion generados dentro de TIMA, y la continuación de uno de ellos, \emph{NetFI-2}, desarrollado en conjunto con TIMA y LCSR cuya vigencia era discutible a causa de las nuevas tecnologías existentes, las cuales demandan cada vez mayores requerimientos de sistema.

Como se planteó anteriormente, a continuación se describirá el ambiente radiactivo y sus consecuencias.