% De radiación en los circuitos integrados
En los últimos años, la industria de los semiconductores ha estado particularmente interesada en los efectos de la radiación en los circuitos integrados, como los Circuitos integrados de aplicaciones específicas ASIC, las matrices de puertas programables de campo FPGA y los sistemas integrados en general ~\cite{Velazco2007}.
La razón detrás de esta motivación radica no solo en el uso de estos sistemas en entornos de radiación hostiles ~\cite{Dixit2011} sino también en el grado creciente de integración de dispositivos integrados en el mismo chip.
En particular, estudios recientes han demostrado que cuanto más miniaturización, mayor es la sensibilidad a los errores inducidos por la radiación ~\cite{Ibe2010}.
Los sistemas embebidos modernos son, por lo tanto, potencialmente susceptibles a partículas de baja energía, como los neutrones y los muones presentes en la atmósfera de la Tierra.

El impacto de las partículas energéticas en los circuitos integrados puede causar alteraciones en el comportamiento de los componentes microelectrónicos.
Estos errores se conocen como efectos de evento único SEE y pueden ser de diferentes tipos.
Entre ellos, los que dan como resultado el cambio de un bit de información en un registro o memoria se denominan Single Event Upsets SEU, mientras que los pulsos transitorios que modifican la lógica combinatoria se conocen como Single Event Transients SETs.
El término Múltiples alteraciones de células MCU se usa para referirse a múltiples errores causados por una sola partícula en posiciones de memoria adyacentes en el chip ~\cite{Quinn2007}.
Cuando las celdas modificadas pertenecen a una misma palabra o registro de memoria, el efecto se denomina Modalidad de bits múltiples MBU.
Cuanto mayor sea la escala de integración, mayor será la probabilidad de que aparezcan MCU y MBU, lo que puede ser un desafío para las técnicas tolerantes a fallas tradicionales, como los códigos de corrección de errores.
En este contexto, existe una creciente necesidad de estimar la sensibilidad de los circuitos integrados modernos.

Para estudiar los efectos de los SEU y los SET en los circuitos digitales, generalmente se realizan pruebas bajo haces de radiación para analizar el comportamiento del dispositivo bajo un flujo de partículas grandes ~\cite{Criswell1984}.
Sin embargo, estas campañas son muy costosas, se basan en la implementación física del Dispositivo bajo prueba DUT y requieren un considerable esfuerzo técnico y programático ~\cite{Velazco2010}.
En consecuencia, las metodologías \emph{simulación} y \emph{emulación} son alternativas cada vez más populares para evaluar y predecir el comportamiento de estos circuitos antes de la fabricación.
En particular, el inyector de fallas NETlist NETFI se propuso en ~\cite{Mansour2013-1} y se extendió en ~\cite{Mansour2013-2} como un método para inyectar fallas en el nivel de transferencia de registro RTL.
Desde la perspectiva del usuario, el lenguaje de descripción de hardware HDL se puede proporcionar convenientemente como entrada mientras no se modifica durante el proceso de inyección de fallas.
En general, NETFI resulta particularmente atractivo ya que combina una buena capacidad de control y observabilidad del experimento con la capacidad de inyectar fallas en un solo ciclo de reloj.
No obstante, esta metodología ha sido criticada por la complejidad y rigidez asociada con el controlador responsable de la ejecución de la inyección de falla ~\cite{Serrano2015}.

En este trabajo, \mbox{NETFI-3} se presenta como una evolución de la metodología NETFI que aborda las debilidades mencionadas anteriormente.

A diferencia de su versión anterior, \mbox{NETFI-3} automatizar de principio a fin el proceso de la campaña de inyección, dejando de lado la enorme cantidad de tiempo que demandaba instanciar un DUT bajo la mejora propuesta en \mbox{NetFI-2} \ref{netfi2}.

\mbox{NETFI-3} se basa en un controlador implementado en un procesador MicroBlaze incorporado, al igual que NetFI-2, pero además incorpora un nuevo núcleo de inyeciones, conocido como MODNET, ahora en su versión 2.0, que realmente permiten a la sistematización y puesta en marcha de campañas de inyección de fallos, reduciendo así la enorme curva de aprendisaje necesaria para realizar pruebas sobre dispositivos de interés a someter a posible radiación, como además reduciendo el tiempo de implementación de manera antes solo imaginada por la metodología NetFI-2.
