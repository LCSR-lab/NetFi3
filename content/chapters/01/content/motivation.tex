A nivel institucional con el presente trabajo  se intenta generar un ambiente de cooperación entre el Laboratorio de Circuitos y Sistemas Robustos (LCSR) de la Facultad de Ciencias Exactas Físicas y Naturales (FCEFyN) de la  Universidad Nacional de Córdoba (UNC) con el laboratorio TIMA. 

A nivel personal con este trabajo se desea terminar la  última etapa como  estudiante de la carrera Ingeniería en Computación de la UNC, y además generar una herramienta que de verdad permita al laboratorio LCSR extender en sus capacidades. Uno de sus aspectos más interesantes es la unión de todo un set de hardware a bajo nivel, con un software de bajo y alto nivel, lo cual presenta un desafío muy atractivo. Durante todo proceso de investigación inicial, por parte de todas las personas que alguna vez participaron en NetFI se sienta una ligera sensación de frustración, debido al alto nivel de complejidad de la metodología. Esto no permite avanzar con nuevas investigaciones, y ha resultado de los avances conseguidos en ser prácticamente olvidados por falta de buenas prácticas en su implementación. 