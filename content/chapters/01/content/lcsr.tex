El Laboratorio de Circuitos y Sistemas Robustos, ``LCSR'', que funciona en el ámbito del Departamento Electrónica de la FCEFyN de la UNC, se enfoco en las siguientes líneas de trabajo principales:
\begin{itemize}

    \item Prevención de Fallas, (PF): Tiene por objetivo la disminución de la probabilidad de ocurrencia de fallas, (remoción o bloqueo de factores causantes de fallas, o insensibilización ante los mismos). Esta línea de trabajo busca el aumento de la confiabilidad del sistema.
    \item Tolerancia a Fallas, (TF): Tiene por objetivo el aumento de la probabilidad y la velocidad de detección y recuperación (y/o adaptación) ante una falla. Esta línea de trabajo busca el aumento de la disponibilidad del sistema.
    \item Escalabilidad de Sistemas, (ES): Tiene por objetivo alcanzar propiedades fundamentales de los sistemas robustos, tales como ``gracefull degradation'', ``gracefull upgradation''. Estas propiedades permiten realizar una suerte de intercambio entre performance y disponibilidad.
Estas líneas de trabajo se pueden enfocar a la electrónica analógica, a la digital, a los sistemas mixtos, (analógicos y digitales), desde las niveles más bajo de abstracción tales como transistores hasta los niveles más altos tales como redes. En resumen a todo tipo de aplicaciones y sistemas electrónicos. A manera de ejemplo se citan a continuación las actuales líneas de investigación del LCSR:
    \item Robustecimiento de Transistores de RF.
    \item Robustecimiento de Procesadores multinúcleos sobre plataformas basadas en ``NOC'', ``Network on Chips'' y redes ``DTN'', ``Delay Tolerant Networks''.
    \item Robustecimiento de sistemas distribuidos sobre la base de redes DTN, ``Delay Tolerant Networks''.
    \item Sistemas de Inyección de fallas para determinar distribuciones de probabilidad y estadísticas de sistemas digitales bajo condiciones de fallas.
\end{itemize}
