El equipo RIS aborda los desafíos fundamentales inducidos por la miniaturización nanométrica cada vez más cotidiana,incluyendo las densidades de fallas muy elevadas causadas por el proceso de fabricación, tensiones de alimentación y temperaturas, el envejecimiento acelerado de los circuitos, las interferencias electromagnéticas y los errores de software como también las restricciones de bajo consumo.

Para contribuir a este desafío, se desarrollan aproximaciones de diseños robustos y de herramientas de certificación en los diferentes niveles de arquitectura de los sistemas, tales como: nivel de circuito, de bloque, microarquitectura, red y software. Los objetivos son múltiples y conciernen al desarrollo y  utilización de  métodos de tolerancia a fallas, autocorrección y autorregulación con el fin de tolerar  fallos de fabricación. En particular aquellos inducidos por variaciones que aparecen durante la vida del sistema.