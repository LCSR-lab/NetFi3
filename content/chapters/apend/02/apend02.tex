\chapter{Codigo tima\_top}\label{aped.B}

\section{tima\_top.vhdl}
    
\begin{lstlisting}
library ieee;
use ieee.std_logic_1164.all;
-----------------------------------------------------

-----------------------------------------------------
entity tima_top is
    port(
        --ad : in std_logic_vector(31 downto 0);
        
         a,d,b, clk, sel: in std_logic;
        s: out std_logic
   );
end entity tima_top;
-----------------------------------------------------

-----------------------------------------------------
architecture arch_tima_top of tima_top is

--------------MUX-----------------
Component mux is
    port (
        a, b, sel: in std_logic;
        x: out std_logic
   );
end component;
-------------------------------

--------------FFD-----------------
Component flipflop is
    port (
        d, clk: in std_logic;
        q: out std_logic
   );
end component;
-------------------------------

-----------SIGNALS--------------------

signal out_and: std_logic;
signal out_ffd: std_logic;
--------------------------------
begin

    mux1: mux
    port map (a => out_ffd, b => out_and, sel => sel, x => s);
    
    flipflop1: flipflop
    --port map (d => ad(0), clk => clk, q => out_ffd);
    port map (d => d, clk => clk, q => out_ffd);
    out_and <= a and b;
    --out_and <= ad(1) and b;

end architecture arch_tima_top;
-----------------------------------------------------

 \end{verbatim}
 \subsection{FF}
 \begin{verbatim}
library ieee;
use ieee.std_logic_1164.all;
-----------------------------------------------------
-----------------------------------------------------
entity flipflop is
    port(
        d, clk: in std_logic;
        q: out std_logic
   );
end entity flipflop;
----------------------------------------------------
-----------------------------------------------------
Architecture arch_flipflop of flipflop is 

begin

    p_ff: process(clk)
    begin
        if(clk'event and clk = '1') then
            q <= d;
        end if;
    end process p_ff;

end architecture arch_flipflop;
-----------------------------------------------------
 \end{verbatim}
    \subsection{MUX}
   \begin{verbatim}

------------------------------------------------
library ieee;
use ieee.std_logic_1164.all;
------------------------------------------------

------------------------------------------------
entity mux is
    port(
        a, b, sel: in std_logic;
        x: out std_logic
   );
end entity mux;
------------------------------------------------

------------------------------------------------
Architecture arch_mux of mux is
begin
    x <= a WHEN sel = '0' ELSE b;

end architecture arch_mux ;
------------------------------------------------


 \end{lstlisting}
   
   


\section{ Netlist tima\_top.vm}
  
\begin{lstlisting}
//
// Written by Synplify Premier 
// Product Version "K-2015.09-SP1"
// Program "Synplify Premier", Mapper "maprc, Build 3057R"
// Thu Jun 23 08:41:45 2016
//
// Source file index table:
// Object locations will have the form <file>:<line>
// file 0 "\/softslin/synplifyvK2015_09_sp1/lib/vhd2008/std.vhd "
// file 1 "\/softslin/synplifyvK2015_09_sp1/lib/vhd/snps_haps_pkg.vhd "
// file 2 "\/softslin/synplifyvK2015_09_sp1/lib/vhd2008/std1164.vhd "
// file 3 "\/softslin/synplifyvK2015_09_sp1/lib/vhd2008/std_textio.vhd "
// file 4 "\/softslin/synplifyvK2015_09_sp1/lib/vhd2008/numeric.vhd "
// file 5 "\/softslin/synplifyvK2015_09_sp1/lib/vhd/umr_capim.vhd "
// file 6 "\/softslin/synplifyvK2015_09_sp1/lib/vhd2008/arith.vhd "
// file 7 "\/softslin/synplifyvK2015_09_sp1/lib/vhd2008/unsigned.vhd "
// file 8 "\/softslin/synplifyvK2015_09_sp1/lib/vhd/hyperents.vhd "
// file 9 "\/tima/solinasm/Downloads/TIMA/flipflop.vhd "
// file 10 "\/tima/solinasm/Downloads/TIMA/mux.vhd "
// file 11 "\/tima/solinasm/Downloads/TIMA/tima_top.vhd "

`timescale 100 ps/100 ps
module mux (
  a_c,
  b_c,
  out_ffd,
  sel_c,
  s_c
)
;
input a_c ;
input b_c ;
input out_ffd ;
input sel_c ;
output s_c ;
wire a_c ;
wire b_c ;
wire out_ffd ;
wire sel_c ;
wire s_c ;
wire GND ;
wire VCC ;
// @10:18
  LUT4 x (
	.I0(a_c),
	.I1(b_c),
	.I2(out_ffd),
	.I3(sel_c),
	.O(s_c)
);
defparam x.INIT=16'h88F0;
  VCC VCC_cZ (
	.P(VCC)
);
  GND GND_cZ (
	.G(GND)
);
endmodule /* mux */

module flipflop (
  out_ffd,
  d_c,
  clk_c
)
;
output out_ffd ;
input d_c ;
input clk_c ;
wire out_ffd ;
wire d_c ;
wire clk_c ;
wire GND ;
wire VCC ;
// @9:21
  FD q_Z (
	.Q(out_ffd),
	.D(d_c),
	.C(clk_c)
);
  VCC VCC_cZ (
	.P(VCC)
);
  GND GND_cZ (
	.G(GND)
);
endmodule /* flipflop */

module tima_top (
  a,
  d,
  b,
  clk,
  sel,
  s
)
;
input a ;
input d ;
input b ;
input clk ;
input sel ;
output s ;
wire a ;
wire d ;
wire b ;
wire clk ;
wire sel ;
wire s ;
wire out_ffd ;
wire GND_x ;
wire VCC_x ;
wire a_c ;
wire d_c ;
wire b_c ;
wire clk_c ;
wire sel_c ;
wire s_c ;
wire clk_ibuf_iso ;
wire GND ;
wire VCC ;
  BUFG clk_ibuf (
	.I(clk_ibuf_iso),
	.O(clk_c)
);
  IBUFG clk_ibuf_iso_cZ (
	.O(clk_ibuf_iso),
	.I(clk)
);
// @11:11
  OBUF s_obuf (
	.O(s),
	.I(s_c)
);
// @11:10
  IBUF sel_ibuf (
	.O(sel_c),
	.I(sel)
);
// @11:10
  IBUF b_ibuf (
	.O(b_c),
	.I(b)
);
// @11:10
  IBUF d_ibuf (
	.O(d_c),
	.I(d)
);
// @11:10
  IBUF a_ibuf (
	.O(a_c),
	.I(a)
);
// @11:44
  mux mux1 (
	.a_c(a_c),
	.b_c(b_c),
	.out_ffd(out_ffd),
	.sel_c(sel_c),
	.s_c(s_c)
);
// @11:47
  flipflop flipflop1 (
	.out_ffd(out_ffd),
	.d_c(d_c),
	.clk_c(clk_c)
);
  assign GND = 1'b0;
  assign VCC = 1'b1;
endmodule /* tima_top */


\end{lstlisting}

\section{Inyecciones en tima\_top.vm}
\begin{lstlisting} 

module tima_top (
  inj,
  a,
  d,
  b,
  clk,
  sel,
  s
)
;
input [0 : 0] inj ;
input a ;
input d ;
input b ;
input clk ;
input sel ;
output s ;
wire a ;
wire d ;
wire b ;
wire clk ;
wire sel ;
wire s ;
wire out_ffd ;
wire GND_x ;
wire VCC_x ;
wire a_c ;
wire d_c ;
wire b_c ;
wire clk_c ;
wire sel_c ;
wire s_c ;
wire clk_ibuf_iso ;
wire GND ;
wire VCC ;

  BUFG clk_ibuf (   .I(clk_ibuf_iso),
    .O(clk_c)
);
  IBUFG clk_ibuf_iso_cZ (   .O(clk_ibuf_iso),
    .I(clk)
);
// @11:11
  OBUF s_obuf (   .O(s),
    .I(s_c)
);
// @11:10
  IBUF sel_ibuf (   .O(sel_c),
    .I(sel)
);
// @11:10
  IBUF b_ibuf (   .O(b_c),
    .I(b)
);
// @11:10
  IBUF d_ibuf (   .O(d_c),
    .I(d)
);
// @11:10
  IBUF a_ibuf (   .O(a_c),
    .I(a)
);
// @11:44
mux mux_uut (
    .a_c(a_c),
    .b_c(b_c),
    .out_ffd(out_ffd),
    .sel_c(sel_c),
    .s_c(s_c)
);
// @11:47
flipflop flipflop_uut (
 .inj(inj[0 : 0]),
    .out_ffd(out_ffd),
    .d_c(d_c),
    .clk_c(clk_c)
);
  assign GND = 1'b0;  assign VCC = 1'b1;endmodule /* tima_top */
\end{verbatim} 
\subsection{ FF}
\begin{verbatim} 
module flipflop (
 inj,
  out_ffd,
  d_c,
  clk_c
)
;

input [0: 0] inj ;
output out_ffd ;
input d_c ;
input clk_c ;
wire out_ffd ;
wire d_c ;
wire clk_c ;
wire GND ;wire VCC ;

// @9:21
  FD_mod q_Z (
 .inj(inj[0]) ,
    .Q(out_ffd),
    .D(d_c),
    .C(clk_c)
);
  VCC VCC_cZ ( .P(VCC)
);
  GND GND_cZ ( .G(GND)
);
endmodule /* flipflop */

\end{verbatim} 
\subsection{ MUX}
\begin{verbatim} 
module mux (
 inj,
  a_c,
  b_c,
  out_ffd,
  sel_c,
  s_c
)
;

input inj ;
input a_c ;
input b_c ;
input out_ffd ;
input sel_c ;
output s_c ;
wire a_c ;
wire b_c ;
wire out_ffd ;
wire sel_c ;
wire s_c ;
wire GND ;wire VCC ;

// @10:18
  LUT4 x ( .I0(a_c),
    .I1(b_c),
    .I2(out_ffd),
    .I3(sel_c),
    .O(s_c)
);
defparam x.INIT=16'h88F0;
  VCC VCC_cZ ( .P(VCC)
);
  GND GND_cZ ( .G(GND)
);
endmodule /* mux */

\end{lstlisting}