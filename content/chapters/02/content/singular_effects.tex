Los \textit{Single Event Effect} (SEE) tienen por origen el mismo fenómeno que aquél de la dosis ionizante, es decir la ionización localizada a lo largo de la trayectoria de la partícula incidente. Ellos resultan principalmente de la acumulación y  colección de cargas en un nodo sensible de circuito. Los iones energéticos que atraviesan este volumen producen directamente una columna de electrones-agujeros que pueden provocar un SEE. Otro mecanismo que puede ionizar la materia y conducir a un SEE se debe a los sub-productos de las colisiones de los protones y  neutrónes transfiriendo toda o una parte de su energía.

Diferentes tipos de eventos pueden producirse como consecuencia del impacto de una partícula ionizante, estos se verán a continuación.