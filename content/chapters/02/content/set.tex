La partícula incidente induce por ionización una cantidad de carga. Esta carga es colectada por el campo eléctrico provocando la creación de una fotocorriente. Las consecuencias de este tipo de efectos son principalmente la generación de pulsos de corriente indeseables que pueden perturbar el funcionamiento de los sistemas digitales y analógicos.

La corriente generada puede propagarse en lógicas combinacionales, 
 fundamentalmente en circuitos integrados con la tecnología CMOS y quizás pueden ser capturados por un elemento de memoria, si es que llegase a ocurrir durante un flanco de reloj. En algunos casos un SET puede resultar en un \textit{Single Event Upset}~\cite{ grandstrand:2004}. 