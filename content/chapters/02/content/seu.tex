La fotocorriente crea un cambio de estado en las estructuras tipo flip-flop (FF). Este fenómeno se produce en el momento en el que la carga inducida por la partícula sobrepasa la cantidad necesaria para cambiar de estado. Este evento no es destructivo, pero puede generar errores perturbando el contenido de las células de memoria (memorias SRAM, FF, registros y todos los circuitos digitales secuenciales).

Este fenómeno también suele llamarse \textit{bit-flip}, puede depender tanto en el instante de ocurrencia como en el tipo de celda donde se produce la perturbación. En efecto, un SEU puede ser tanto silencioso (tocando una celda de memoria que no es utilizada) o puede producir errores críticos.

La sensibilidad de un SEU en los dispositivos electrónicos varía enormemente de acuerdo a la tecnología y a varios otros parámetros. En particular, la reducción del tamaño de los transistores o su corriente de alimentación tienden a decrecer las cargas críticas, por consiguiente incrementa la sensibilidad a los SEEs ~\cite{ grandstrand:2004}.