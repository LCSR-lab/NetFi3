Dentro del ambiente espacial, las perturbaciones causadas en los circuitos integrados dependen del tipo de radiación incidente. Normalmente el tipo de radiación puede clasificarse según las fuentes que las causan y por el medio en el que se transmiten: electrones, protones, iones y  energías diversas ~\cite{boudenot1995}. De tal manera que según quién y cómo se emane la partícula se definirá las posibles consecuencias causadas y su probabilidad de ocurrencia. 

El ambiente radiactivo espacial se compone principalmente de las siguientes fuentes radiactivas según sus orígenes: la radiación cósmica \textit{Global Cosmic Rays } (GCR), el viento solar, las irrupciónes solares y los cinturones de radiación ~\cite{stapor1988} ~\cite{tang2004semm}.

\subsubsection{La radiación cósmica}
Su origen se desconocía hasta que las investigaciones realizadas por V.Hess en 1912~\cite{bourrieau1991environnement} arrojaron que es parte de una radiación proveniente de fuentes galácticas y extragalácticas. Está constituida de protones (87\%), de helio (12\%) y de iones pesados (1\%) y pueden medirse energías  superiores a 1 GeV, llegando hasta los $10^{11}$ GeV ~\cite{fleischer1975nuclear}. 

Dentro del sistema solar, el flujo que caracteriza la población de iones cósmicos está determinado por el ciclo de actividad solar: el viento solar se opone al flujo de rayos cósmicos cuando el sol está en un periodo de máxima actividad. Las radiaciones son entonces muy energéticas pero los flujos asociados son relativamente débiles. Es necesario tomar esto en cuenta en casos donde las misiones espaciales son largas (de varios años) ya que la probabilidad de aparición de un evento potencialmente destructivo es alto. 