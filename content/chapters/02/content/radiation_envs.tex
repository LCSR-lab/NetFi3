Nuestro sistema planetario está compuesto de una estrella y de cuerpos gravitando alrededor de ella. El sol situado en el seno de nuestra galaxia, constituye con los rayos cósmicos el origen de todas las radiaciones. Cada segundo, su centro caliente fusiona aproximadamente 700 millones de toneladas de hidrógeno y produce 695 millones de toneladas de helio y 5 millones de toneladas de energía bajo la forma de rayos gamma ~\cite{eddy2009}. Esto explica la energía emanada por el sol que es aproximadamente 368 millones de Megawatts.
   

La superficie del sol llamada fotósfera tiene una temperatura cercana a los 5500\grad C. Por encima de la fotósfera reposa una pequeña capa llamada la cromósfera. Por encima de la cromósfera se encuentra la corona. Una región pobre en elementos, que se extiende por millones de kilómetros sobre el espacio. La cordona es solo   visible durante un eclipse  y su temperatura suele sobrepasar un millón de grados Celsius. El campo magnético del sol es muy potente en comparación al terrestre. Su magnetósfera llamada también heliósfera se extiende mucho más allá que Plutón ~\cite{eddy2009}.

Existen estudios  realizados sobre la radiación solar, las consecuencias en su entorno y su interacción con los circuitos integrados. Comenzaron a finales de los 70 y muestran que los componentes electrónicos pueden perturbarse por consecuencia de las radiaciones del ambiente en los cuales operan. En 1975, Binder presentó la primer publicación que menciona el rol probable de la radiación sobre los circuitos que operan  en satélites \cite{4328188}. En 1979 las investigaciones realizadas  por Ziegler y Lanford concluyeron   que las radiaciones juegan un rol  importante en las alturas aviónicas y que las perturbaciones que vienen de las reacciones \textit{neutro-silicium} se deben considerar en el desarrollo de circuitos y sistemas destinados a operar en grandes alturas ~\cite{ziegler1996}. 
El desarrollo de la tecnología  \textit{Very-Large-Scale Integration} (VLSI) tuvo consecuencias directas sobre la sensibilidad de los circuitos ante perturbaciones y errores causados por la radiación. Por lo tanto, esta temática  debe ser tomado muy en cuenta para asegurar el buen funcionamiento de sistemas sin importar dónde ellos se desenvuelven, en el espacio, en la atmósfera y posiblemente en la tierra también.